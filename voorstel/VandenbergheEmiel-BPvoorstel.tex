%==============================================================================
% Sjabloon onderzoeksvoorstel bachproef
%==============================================================================

\documentclass{hogent-article}

\usepackage{pgfgantt}
\usepackage{microtype}

\addbibresource{voorstel.bib}

\studyprogramme{Professionele bachelor toegepaste informatica}
\course{Bachelorproef}
\assignmenttype{Onderzoeksvoorstel}

\academicyear{2025-2026}

\title{Observability voor Multi-Site Live-Streamingomgevingen:
Een Proof-of-Concept voor Metrics-, Log- en Flowmonitoring binnen Mediaventures}

\author{Emiel Vandenberghe en Ruben Van Bruysel}
\email{emiel.vandenberghe3@student.hogent.be en ruben.vanbruyssel@student.hogent.be}

\projectrepo{https://github.com/HoGentTIN/paper-research-methods-nl-24-25-rm\_van\_bruyssel\_vandenberghe}

\supervisor[Co-promotor]{Arne Van Renthergem (Mediaventures)}

\specialisation{System \& Network Administrator}
\keywords{Observability, Peplink, NDI}

\begin{document}

\begin{abstract}
Mediaventures is een audiovisuele firma die gespecialiseerd is in livestreamingsdiensten, zaalbouw en -inrichting voor internationale medische congressen zoals het \href{https://www.linkforwoundhealing.info/en/events/2025/link-for-wound-healing-congress-2025}{LINC Congress} en het \href{https://www.escardio.org/Congresses-Events/ESC-Congress}{ESC Congress}. Tijdens hun projecten worden verschillende live-surgery-locaties en venues met een centrale site verbonden om hoogwaardige video- en audiostreams in real\-time uit te wisselen. Deze samenwerking steunt op een complexe mix van netwerk\-ap\-pa\-ra\-tuur en verbindingen, waardoor de kwaliteit van de livestream door veel verschillende factoren tegelijk beïnvloed wordt. Vandaag bestaat er echter geen centraal systeem dat deze volledige keten bewaakt en vroegtijdig aangeeft wanneer er problemen ontstaan. Daardoor moeten techniekers op elk toestel afzonderlijk inloggen om fouten op te sporen, wat tijdrovend is en het risico op kwaliteits\-verlies verhoogt.

Dit onderzoek onderzoekt hoe een geïntegreerde monitorings\-op\-los\-sing kan helpen om deze diverse infrastructuur end-to-end op te volgen en sneller inzicht te krijgen in de gezondheid en prestaties van alle betrokken systemen.
\end{abstract}

% Toon alleen hoofdsecties in de inhoudstafel (geen subsecties)
\setcounter{tocdepth}{1}

\tableofcontents

%---------- Inleiding ---------------------------------------------------------

\section{Inleiding}
\label{sec:inleiding}

Tijdens live-surgeryprojecten en medische congressen maakt Mediaventures gebruik van een uitgebreid audiovisueel netwerk waarin meerdere locaties gelijktijdig met elkaar communiceren. Een typisch scenario is dat een chirurgische ingreep live wordt gestreamd vanuit een operatiezaal — bijvoorbeeld in Duitsland — naar een congreszaal in Leipzig, terwijl dezelfde beelden ook worden doorgestuurd naar de centrale site in Bornem. Afhankelijk van het project verlopen deze verbindingen soms rechtstreeks tussen locaties, en soms via Bornem als tussenknooppunt. Deze flexibiliteit is essentieel, maar maakt de totale infrastructuur ook bijzonder complex.

De omgeving bestaat uit een mix van routers, switches, 5G- en satellietverbindingen, NDI-netwerken en verschillende AV-systemen die elk hun eigen rol vervullen. Omdat al deze componenten nauw samenwerken, kan een probleem op één plaats meteen gevolgen hebben voor alle verbonden locaties. Storingen in bandbreedte, latency, packet loss of lokale netwerkbelasting kunnen de kwaliteit van de livestream merkbaar aantasten. Vandaag bestaat er echter geen centraal platform dat de volledige datastromen, toestellen en verbindingen gezamenlijk bewaakt. Hierdoor moeten techniekers nog steeds handmatig inloggen op afzonderlijke apparaten om problemen te analyseren, wat foutdetectie vertraagt en het risico op kwaliteitsverlies vergroot.

\subsection{Centrale onderzoeksvraag}

\textbf{Hoe kan Mediaventures een geïntegreerde monitoring- en observabilityoplossing inzetten om de volledige audiovisuele infrastructuur end-to-end op te volgen en storingen sneller te detecteren en te interpreteren?}

\subsection{Deelvragen}

Om de centrale onderzoeksvraag te beantwoorden, worden de volgende deelvragen onderzocht:

\begin{enumerate}
\item \textbf{Deelvraag voor het probleemdomein}: Welke soorten data en statusinformatie zijn beschikbaar op de verschillende netwerk- en AV-com\-po\-nen\-ten binnen deze internationale workflow, en welke parameters zijn cruciaal om kwaliteitsproblemen vroegtijdig te detecteren?

\item \textbf{Deelvraag voor het oplossingsdomein}: Welke observability- en monitoringtechnieken sluiten het best aan bij audiovisuele producties met hoge betrouwbaarheidseisen, en welke bestaande tools zijn compatibel met de infrastructuur van Mediaventures?

\end{enumerate}

\subsection{Doelstellingen}

Het probleem- en oplossingsdomein bevindt zich op het snijpunt van AV-productie, netwerktechnologie en realtime kwaliteitsbewaking. Dit onderzoek heeft de volgende concrete doelstellingen:

\begin{itemize}
\item Een onderbouwde aanbeveling formuleren voor een observabilityoplossing die aansluit bij de specifieke noden van Mediaventures
\item Een functioneel proof-of-concept ontwikkelen dat metrics, logs en netwerkflows integreert in één gecentraliseerd dashboard
\item Aantonen hoe deze oplossing de tijd tot detectie van storingen kan verkorten en de troubleshooting-efficiëntie kan verhogen
\item Een repliceerbare testmethodologie opleveren waarmee verschillende monitoringtools objectief vergeleken kunnen worden
\end{itemize}

De methodologische aanpak omvat literatuurstudie, een analyse van beschikbare protocollen en datastromen, een toolvergelijking en de opzet van een virtuele testomgeving. De rest van dit document bespreekt eerst de theoretische achtergrond, daarna de methode, gevolgd door de verwachte resultaten en conclusies.

%---------- Literatuurstudie ---------------------------------------------------

\section{Literatuurstudie}
\label{sec:literatuurstudie}

\subsection{Observability in audiovisuele netwerkarchitecturen}

In audiovisuele productieomgevingen is observability een essentieel concept om de betrouwbaarheid en prestaties van complexe ketens te kunnen beoordelen. Videostreams zijn gevoelig voor vertraging en verlies van datapakketten, waardoor afwijkingen in de netwerklaag onmiddellijk zichtbare impact hebben op de uiteindelijke weergave. Dit geldt in het bijzonder voor live-omgevingen waarin meerdere geografisch gescheiden locaties gelijktijdig video en audio uitwisselen. \textcite{Tommasi2021} tonen aan dat real-time monitoring van parameters zoals latency en packet loss cruciaal is om degradatie in streamingkwaliteit tijdig te detecteren. De literatuur benadrukt daarbij dat observability niet beperkt mag blijven tot netwerkmetingen, maar verschillende niveaus moet combineren om complexe audiovisuele workflows correct te interpreteren.

\subsection{NDI als intern audiovisueel transportprotocol}

NDI (Network Device Interface) is een AV-over-IP-technologie die ontworpen is voor situaties waarin meerdere videobronnen binnen een lokaal netwerk (LAN) beschikbaar moeten zijn. Volgens de officiële NDI-documentatie maakt het protocol gebruik van hoge bandbreedtes en lage latency om videobronnen vrijwel in realtime tussen apparaten te transporteren \autocite{NewTek2022}. NDI-apparaten kunnen elkaar automatisch ontdekken binnen hetzelfde subnet, waardoor videobronnen flexibel kunnen worden gerouteerd zonder bijkomende bekabeling.

Deze voordelen brengen ook vereisten met zich mee: NDI-prestaties hangen sterk af van netwerkcondities, vooral bandbreedte, multicastondersteuning en correcte VLAN-configuratie. Deze afhankelijkheden zijn uitgebreid gedocumenteerd door \textcite{NewTek2022} en sluiten aan bij observaties uit de sector. Binnen Mediaventures worden NDI-bronnen doorgaans samengebracht op een softwarematige videomixer die meerdere inkomende NDI-streams verwerkt tot één samengestelde output \autocite{VanRenthergem2025}. Deze stap vormt een kritisch punt in de keten omdat zowel netwerkcongestie als processorbelasting op dit moment onmiddellijk invloed heeft op de latere transmissiestroom.

\subsection{De omzetting van NDI naar SRT voor transport over grotere afstanden}

Omdat NDI ontworpen is voor gebruik binnen een LAN en niet geschikt is voor transmissie over grote afstanden, wordt de samengestelde NDI-output lokaal omgezet naar SRT (Secure Reliable Transport). Haivision, de organisatie achter SRT, beschrijft het protocol als een oplossing die storingen in internetverbindingen compenseert door middel van foutcorrectie en jitterbuffers \autocite{Haivision2023}. Tijdens dit omzettingsproces wordt de NDI-stream gecodeerd naar een geschikt videocompressieformaat en verpakt in een SRT-stream.

Deze omzetting vormt een belangrijke overgang van LAN-gebaseerde videodistributie naar WAN-transmissie. Storingen in de NDI-laag — zoals haperingen of te hoge belasting — worden op dit punt direct doorgegeven aan de SRT-encoder, die ze niet altijd kan corrigeren. Dit maakt monitoring op deze overgang van bijzonder belang \autocite{VanRenthergem2025}.

\subsection{De omzetting van SRT terug naar een lokaal formaat op de ontvangende locatie}

Wanneer de SRT-stream aankomt op de venue of in Bornem, wordt deze opnieuw gedecodeerd. De ontvangende software zet de stream terug om in een videobron die binnen het lokale netwerk kan worden gebruikt. Volgens \textcite{Haivision2023} ondersteunt SRT dit decodeerproces zonder dat de oorspronkelijke kwaliteit onnodig wordt aangetast. In veel audiovisuele workflows wordt de gedecodeerde stream opnieuw als NDI beschikbaar gemaakt zodat deze lokaal op dezelfde manier kan worden gerouteerd als andere videobronnen. Dit zorgt voor een consistente end-to-end workflow waarin NDI wordt gebruikt voor interne distributie en SRT voor transmissie over grotere afstanden \autocite{VanRenthergem2025}. Deze architectuur is in lijn met gangbare AV-over-IP-praktijken.

\subsection{Peplink, PepVPN en de rol van multi-WAN-routing}

De WAN-architectuur die de verschillende locaties verbindt, wordt in de onderzochte Mediaventures-workflow gerealiseerd met Peplink-routers. Peplink beschrijft PepVPN als een technologie die meerdere soorten internetverbindingen — zoals 5G, Starlink of vaste lijnverbindingen — kan combineren tot één redundante VPN-tunnel \autocite{Peplink2023}. Deze tunnel verdeelt het verkeer over de beschikbare verbindingen en kan dynamisch omschakelen naar alternatieve paden wanneer één verbinding tijdelijk uitvalt. Dit maakt de technologie bijzonder geschikt voor audiovisuele productietoepassingen waarin continue datastromen vereist zijn.

De live-surgerylocaties gebruiken doorgaans Peplink 20X-routers die via USB-modems toegang kunnen krijgen tot 5G-netwerken. Venues maken gebruik van krachtigere Peplink 380X-routers die hogere bandbreedtes ondersteunen. In bepaalde configuraties wordt het verkeer via de centrale site in Bornem geleid, die beschikt over een stabiele glasvezelverbinding en daardoor als tussenknooppunt functioneert. Dit routingmodel is bevestigd vanuit de praktijkervaring binnen Mediaventures \autocite{VanRenthergem2025} en sluit aan bij de technische opties die Peplink zelf documenteert.

\subsection{InControl2 als platform voor routerobservatie}

InControl2 is het cloudplatform waarmee Peplink beheerders toelaat om routers centraal te monitoren. Volgens de officiële documentatie biedt het platform inzicht in WAN-status, historische latenties, VPN-kwaliteit en configuratiebeheer \autocite{Peplink2023}. Hoewel InControl2 waardevolle routerinformatie biedt, vormt het geen complete observabilityoplossing. Het platform monitort geen audiovisuele protocollen zoals NDI en SRT, en integreert evenmin gegevens van andere netwerkcomponenten. Deze beperking wordt ook bevestigd in secundaire literatuur die benadrukt dat single-vendor monitoringplatformen onvoldoende zicht bieden op heterogene AV-netwerken \autocite{Elradi2025}.

\subsection{Virtuele onderzoekomgeving via FusionHub}

FusionHub, de virtuele router van Peplink, maakt het mogelijk om PepVPN-verbindingen te simuleren in cloudomgevingen zoals AWS of Google Cloud. De officiële documentatie beschrijft hoe een FusionHub-instantie kan worden ingezet om gedistribueerde netwerken virtueel te verbinden en te testen \autocite{Peplink2023}. Binnen onderzoekscontext maakt dit het mogelijk om realistische datastromen te reproduceren en fluctuaties in bandbreedte of latency te simuleren, zonder afhankelijk te zijn van fysieke installaties of liveproducties. De toepassing van FusionHub in een onderzoeksopstelling met twee live-surgerylocaties, een venue en Bornem is gebaseerd op praktijkervaring binnen Mediaventures \autocite{VanRenthergem2025}.

\subsection{Synthese}

Het geheel van de literatuur toont dat audiovisuele workflows meerdere gespecialiseerde protocollen, netwerkcomponenten en afhankelijkheden combineren. NDI presteert optimaal binnen LAN-omgevingen, terwijl SRT ontworpen is voor WAN-transport. Peplink’s PepVPN biedt een robuuste basis voor multi-WAN-routing tussen locaties. Omdat geen enkel bestaand platform al deze lagen tegelijk monitort, is een geïntegreerde observabilityaanpak noodzakelijk. Zowel onderzoeksbronnen als praktijkervaring ondersteunen de conclusie dat een gecombineerde analyse van netwerk- en AV-parameters essentieel is om betrouwbare end-to-end prestaties te garanderen.


%---------- Evaluatie van Monitoring-Tools ------------------------------------------------------

\section{Evaluatie en selectie van monitoringtools}
\label{sec:evaluatie}

Het ontwikkelen van een observability\-op\-los\-sing voor een audiovisuele netwerk\-ar\-chi\-tec\-tuur vereist monitoring die zowel netwerkparameters als AV-specifieke transportlagen kan interpreteren. Om tot een geschikte oplossing te komen, werden drie open-source monitorings\-op\-los\-sin\-gen geëvalueerd: de PLG-stack (Prometheus, Loki en Grafana), ntopng/nProbe en de OpenTelemetry Collector. Deze evaluatie is gebaseerd op documentatie, literatuur en technische bronnen over de werking en integratie\-mo\-ge\-lijk\-he\-den van deze systemen.

\subsection{Prometheus, Loki en Grafana (PLG-stack)}

Prometheus is een open-source metricscollector die ontworpen is voor het ophalen van tijdreeks\-data uit API's, SNMP-interfaces en custom exporters (Prometheus Authors, 2023). Omdat InControl2 van Peplink een REST-API aanbiedt, kunnen router\-sta\-tis\-tie\-ken zoals latency, traffic counters en VPN-ge\-zond\-heid eenvoudig worden opgehaald en opgeslagen in Prometheus. Loki is een schaalbaar log\-ag\-gre\-ga\-tie\-plat\-form dat logs indexeert op labels in plaats van op inhoud, wat het efficiënter maakt voor gedistribueerde systemen (Grafana Labs, 2023). Grafana fungeert als visualisatie\-laag en kan zowel Prometheus-metrics als Loki-logs combineren in één dashboard.

Deze toolstack is bijzonder geschikt voor het correleren van netwerk\-ge\-beur\-te\-nis\-sen met AV-telemetrie. SRT-encoders kunnen bijvoorbeeld jitter, resend-rates of buffer\-sta\-tis\-tie\-ken via exporters beschikbaar maken, terwijl NDI-mixersoftware CPU-belasting en performance-metingen kan blootstellen via API's of custom agents. Het belangrijkste nadeel van PLG is dat Prometheus geen direct inzicht biedt in netwerkflows; het meet uitsluitend waarden die door systemen zelf worden geëxporteerd.

\subsection{ntopng/nProbe voor netwerkflowanalyse}

ntopng is een open-source netwerkmonitor die zich richt op diepgaande analyse van netwerkflows. Volgens de officiële documentatie ondersteunt ntopng NetFlow, sFlow en IPFIX-export om netwerkstromen te analyseren op basis van bron, bestemming, protocol, VLAN en tijdspatroon \autocite{ntop2023}. In tegenstelling tot Prometheus inspecteert ntopng geen counters maar de daadwerkelijke verkeersstromen. Dit is cruciaal in audiovisuele workflows waar NDI honderden megabits per seconde genereert en SRT afhankelijk is van stabiele, burst-vrije transportpaden.

Peplink-routers ondersteunen native NetFlow en IPFIX-export, waardoor ze rechtstreeks meetdata kunnen aanleveren aan ntopng of nProbe \autocite{Peplink2023}. Hierdoor kan ntopng inzicht bieden in welke specifieke stromen verant\-woor\-de\-lijk zijn voor band\-breed\-te\-pie\-ken, congestie, multicast\-floo\-ding of packet loss. Dit maakt ntopng complementair aan PLG: waar PLG toont dat een router 400 Mbps gebruikt, toont ntopng welke stream daarvoor verantwoordelijk is en wanneer burst\-ge\-drag of pad\-wij\-zi\-gin\-gen optreden.

\subsection{OpenTelemetry Collector}

OpenTelemetry is een open-source observability\-frame\-work dat gestandaardiseerde verzameling van metrics, logs en traces mogelijk maakt. De OpenTelemetry Collector fungeert als een vendor-neutrale datalaag die observability\-ge\-ge\-vens kan ontvangen, converteren en doorsturen naar verschillende backends zoals Grafana, Prometheus of Elastic \autocite{CNCF2023}. Hoewel dit veel flexibiliteit biedt, vereist OTel meer configuratie en biedt het geen eigen visualisatie. Daarom is OTel vooral geschikt als toekomst\-ge\-rich\-te uit\-brei\-dings\-laag, maar niet als primaire monitoringtool voor dit onderzoek.

\subsection{Vergelijkende analyse}

Uit de vergelijking blijkt dat elk platform een ander deel van het observability\-do\-mein bestrijkt. De PLG-stack biedt de beste ondersteuning voor het verzamelen van metrics en logs en kan eenvoudig worden geïntegreerd met zowel Peplink's InControl2 API als AV-software. ntopng vult PLG aan door inzicht te geven in de netwerk\-stro\-men zelf---informatie die onmisbaar is voor AV-workflows maar ontbreekt bij traditionele metrics\-sys\-te\-men. De OpenTelemetry Collector biedt waarde als standaardisatie- en uitbreidings\-mecha\-nis\-me, maar is niet noodzakelijk binnen de proof-of-concept.

Door de PLG-stack te combineren met ntopng ontstaat een observability\-ar\-chi\-tec\-tuur die zowel breed als diep is: PLG geeft het overzicht en correlatie\-mo\-ge\-lijk\-he\-den, terwijl ntopng het detailniveau levert om audiovisuele verkeers\-pa\-tro\-nen te analyseren.

\subsection{Gekozen oplossing}

Op basis van de bovenstaande analyse wordt de PLG-stack gekozen als centrale monitorings\-op\-los\-sing, aangevuld met ntopng als gespecialiseerde flow-analysetool. De PLG-stack wordt geselecteerd vanwege haar open-source karakter, flexibiliteit, integratie\-mo\-ge\-lijk\-he\-den en sterke visualisatie\-ca\-pa\-ci\-tei\-ten. ntopng wordt toegevoegd omdat AV-over-IP-verkeer zich niet laat reduceren tot simpele metrics; diepgaande flowanalyse is noodzakelijk om NDI-band\-breed\-te\-pie\-ken, multicast\-ge\-drag, SRT-transport\-kwa\-li\-teit en WAN-congestie te kunnen begrijpen.

Samen vormen deze tools een robuust en volledig observability\-plat\-form dat geschikt is voor zowel de virtuele test\-op\-stel\-ling (met FusionHub en meerdere gesimuleerde locaties) als de operationele Mediaventures-workflow.


%---------- Methodologie ------------------------------------------------------
\section{Methodologie}
\label{sec:methodologie}

Het onderzoek volgt een toegepaste onderzoeksopzet die bestaat uit vijf opeenvolgende fasen. Elke fase bouwt voort op de resultaten van de voorgaande en levert concrete output die nodig is voor de volgende stap. De planning is afgestemd op de officiële bachelorproeftijdslijn, met de draft-deadline op 2 maart 2026 en de finale indiening op 29 mei 2026.

\subsection{Fase 1: Analyse van de Mediaventures-infrastructuur (februari 2026, weken 1--3)}

De eerste fase start bij aanvang van semester 2 en richt zich op het in kaart brengen van de bestaande netwerk- en streamingarchitectuur bij Mediaventures. Dit gebeurt via drie complementaire methoden. Ten eerste worden configuratiebestanden en netwerkdiagrammen verzameld van de Peplink 20X- en 380X-routers, de Netgear AV-Line M4250-switches en de NDI-softwaremixers. Ten tweede vinden semigestructureerde interviews plaats met de technische medewerkers, waaronder co-promotor Arne Van Renthergem, om impliciete kennis over typische workflows en bekende knelpunten te documenteren. Ten derde wordt exploratieve dataverzameling uitgevoerd tijdens een live-project om te observeren welke SNMP-OID's, Syslog-berichten en NetFlow-exports effectief beschikbaar zijn op de verschillende apparaten.

De output van deze fase is een gedetailleerd overzicht van alle databronnen die binnen de infrastructuur kunnen worden gemonitord, inclusief de protocollen en interfaces waarmee deze data toegankelijk is. Dit overzicht vormt de basis voor de selectie van geschikte monitoringtools in fase 3.

\subsection{Fase 2: Literatuurstudie (februari--maart 2026, weken 2--5)}

Parallel aan de infrastructuuranalyse wordt een systematische literatuurstudie uitgevoerd. De zoekstrategie richt zich op vier thema's: observability in audiovisuele en latency-gevoelige netwerken, QoS-monitoring in streaming workflows, metrics- en logging-architecturen voor gedistribueerde systemen, en netwerksecurity- en eventanalyse. Bronnen worden gezocht via Google Scholar, IEEE Xplore en de ACM Digital Library met zoektermen zoals "AV-over-IP monitoring", "NDI network performance", "SRT quality metrics" en "multi-site streaming observability".

De literatuurstudie levert een theoretisch kader op dat richting geeft aan de selectie van relevante tools, protocollen en architectuurmodellen. Daarnaast identificeert deze fase welke parameters volgens de wetenschappelijke literatuur het meest indicatief zijn voor kwaliteitsproblemen in live-streamingomgevingen. De resultaten van deze fase worden verwerkt in de draft die op 2 maart 2026 wordt ingediend.

\subsection{Fase 3: Vergelijkende studie van monitoringtools (maart 2026, weken 5--8)}

In de derde fase worden de geselecteerde monitoringtools systematisch vergeleken. Op basis van de literatuurstudie en de infrastructuuranalyse worden vijf evaluatiecriteria gehanteerd: verwerkingstijd tussen binnenkomende data en visualisatie, volledigheid en betrouwbaarheid van metrics, logs en flows, schaalbaarheid bij toenemende eventbelasting, integratiemogelijkheden met de Peplink-, NDI- en Netgear-apparatuur, en resourcegebruik en onderhoudscomplexiteit.

De vergelijking gebeurt op basis van documentatieanalyse, technische specificaties en kleinschalige tests met gesimuleerde data. Elke tool wordt beoordeeld op een vijfpuntsschaal per criterium. De resultaten worden samengevat in een vergelijkingsmatrix die de sterke en zwakke punten van elke oplossing visualiseert. In de periode van 2 tot 20 maart vindt tevens de mondelinge toelichting aan de promotor plaats over de literatuurstudie en methodologie.

\subsection{Fase 4: Opbouw virtuele testomgeving (maart--april 2026, weken 7--11)}

Omdat de productieomgeving van Mediaventures niet continu beschikbaar is voor tests, wordt een virtuele testomgeving opgezet. Deze omgeving simuleert de multi-site architectuur met behulp van FusionHub-instanties die de Peplink-routers representeren. Twee FusionHub-instanties fungeren als live-surgerylocaties, één als venue en één als centrale site in Bornem. De instanties worden verbonden via PepVPN-tunnels over het publieke internet, waardoor realistische WAN-condities ontstaan.

Binnen deze virtuele omgeving worden traffic generators ingezet om NDI-, SRT- en generieke UDP/TCP-profielen te simuleren. Tools zoals iperf3 en ffmpeg genereren datastromen met variabele bitrates en burstpatronen die representatief zijn voor audiovisueel verkeer. De Netgear-switches worden waar mogelijk fysiek opgenomen in de testopstelling; indien dit niet haalbaar is, worden hun SNMP- en NetFlow-exports gesimuleerd via softwarematige alternatieven.

\subsection{Fase 5: Proof-of-concept implementatie en validatie (april--mei 2026, weken 10--15)}

In de laatste fase wordt de gekozen observabilityarchitectuur — de PLG-stack aangevuld met ntopng — geïmplementeerd in de virtuele testomgeving. De implementatie omvat de configuratie van Prometheus-exporters voor de InControl2-API en de SRT-encoders, de opzet van Loki voor logaggregatie van Syslog-berichten, de integratie van ntopng voor NetFlow-analyse en de creatie van Grafana-dashboards die alle databronnen combineren.

De validatie gebeurt via drie testscenario's: een baseline-scenario zonder netwerkstoringen, een scenario met gesimuleerde packet loss en latencypieken, en een scenario met bandbreedtecongestie door concurrerende datastromen. Per scenario wordt gemeten hoe snel de monitoringoplossing afwijkingen detecteert en hoe accuraat de gerapporteerde metrics overeenkomen met de werkelijke netwerkcondities. De resultaten worden vergeleken met de huidige situatie waarin techniekers handmatig moeten inloggen op afzonderlijke apparaten. De finale draft wordt ingediend op 4 mei 2026, waarna de laatste weken worden besteed aan het finaliseren van de bachelorproef voor de definitieve indiening op 29 mei 2026.

\subsection{Planning}

De volledige planning van het onderzoek over twee semesters wordt weergegeven in het Gantt-diagram in figuur~\ref{fig:gantt}. Het diagram toont de tijdslijn vanaf de voorbereidingsfase in semester 1 tot en met de definitieve indiening in semester 2, inclusief alle belangrijke mijlpalen en deadlines.

\begin{figure}[h]
\centering
\includegraphics[width=\linewidth,height=0.3\textheight,keepaspectratio]{mermaid_bap.png}
\caption{Gantt-diagram van de onderzoeksfasen over twee semesters.}
\label{fig:gantt}
\end{figure}

\subsection{Risico's en reflectie}

Bij de uitvoering van dit onderzoek zijn verschillende potentiële risico's geïdentificeerd waarvoor mitigerende maatregelen zijn voorzien:

\textbf{Beschikbaarheid testinfrastructuur}: De productieomgeving van Mediaventures is niet continu beschikbaar voor uitgebreide tests. Dit risico wordt gemitigeerd door de ontwikkeling van een virtuele testomgeving met FusionHub. Deze aanpak maakt het mogelijk om reproduceerbare tests uit te voeren zonder afhankelijk te zijn van de beschikbaarheid van fysieke apparatuur tijdens live-projecten.

\textbf{Complexiteit multi-vendor integratie}: De infrastructuur bestaat uit componenten van verschillende leveranciers (Peplink, Netgear, diverse NDI- en SRT-apparaten) die elk hun eigen managementprotocollen gebruiken. Om dit te adresseren wordt in fase 1 uitgebreid vooronderzoek gedaan naar de beschikbare API's en exportmogelijkheden. Indien bepaalde componenten geen standaard exportformaten ondersteunen, zullen custom exporters worden ontwikkeld of alternatieve dataverzamelingsmethoden worden onderzocht.

\textbf{Beperkte toegang tot vergelijkbare onderzoekscontexten}: Audiovisuele livestreaming voor medische congressen is een nichemarkt met beperkte publieke documentatie. Dit wordt gecompenseerd door nauwe samenwerking met de co-promotor van Mediaventures, die praktijkervaring en inzicht in de specifieke uitdagingen kan verschaffen. Daarnaast wordt de literatuurstudie breder getrokken naar aanverwante domeinen zoals IP-video surveillance, live broadcasting en enterprise network monitoring.

\textbf{Tijdsmanagement}: Het onderzoek omvat zowel theoretische analyse als praktische implementatie binnen een beperkt tijdsbestek. Om dit te beheersen is een gedetailleerde planning opgesteld met duidelijke mijlpalen per fase. Regelmatige voortgangsbesprekingen met de promotor en co-promotor zorgen voor tijdige feedback en bijsturing indien nodig. De keuze voor open-source tools met actieve communities vermindert bovendien het risico op langdurige technische blokkades.

Deze risico's zijn bewust geïdentificeerd tijdens de voorbereidingsfase, en de voorgestelde mitigaties zijn realistisch haalbaar binnen de beschikbare tijd en middelen. De reflectie op deze uitdagingen toont dat het onderzoek zorgvuldig is voorbereid en dat er alternatieven beschikbaar zijn bij onvoorziene complicaties.

%---------- Verwachte resultaten ----------------------------------------------

\section{Verwacht resultaat, conclusie}
\label{sec:verwachte_resultaten}

Op basis van de literatuurstudie en de toolvergelijking wordt verwacht dat de combinatie van de PLG-stack met ntopng een effectieve observabilityoplossing vormt voor de audiovisuele infrastructuur van Mediaventures. De PLG-stack biedt een flexibel platform voor het verzamelen en visualiseren van metrics en logs, terwijl ntopng het detailniveau levert dat nodig is om specifieke netwerkstromen en hun gedrag te analyseren.

De proof-of-concept zal naar verwachting aantonen dat gecentraliseerde monitoring de tijd tussen het ontstaan van een probleem en de detectie ervan aanzienlijk verkort. Waar techniekers momenteel handmatig moeten inloggen op afzonderlijke apparaten om de oorzaak van kwaliteitsverlies te achterhalen, zal het observabilityplatform deze informatie proactief presenteren via dashboards en alerting. Concreet wordt verwacht dat de detectietijd voor veelvoorkomende problemen zoals packet loss, latencypieken en bandbreedtecongestie met minstens 50 procent afneemt ten opzichte van de huidige werkwijze.

Daarnaast zal het onderzoek inzicht opleveren in welke parameters het meest indicatief zijn voor kwaliteitsproblemen in AV-over-IP-omgevingen. De verwachting is dat een combinatie van VPN-latency, jitterbufferstatistieken van SRT-encoders en NetFlow-data over NDI-multicastverkeer samen het beste voorspellend vermogen biedt. Deze parameters zullen worden opgenomen in de Grafana-dashboards als primaire indicatoren.

Tot slot wordt verwacht dat de virtuele testomgeving met FusionHub een reproduceerbare basis biedt voor verdere ontwikkeling. De configuraties, dashboards en alertingregels die tijdens het onderzoek worden ontwikkeld, kunnen na afloop worden overgedragen aan Mediaventures voor integratie in hun operationele workflows. De meerwaarde voor het bedrijf bestaat dus niet alleen uit een aanbeveling, maar ook uit een werkend prototype dat als startpunt dient voor productie-implementatie.


\printbibliography[heading=bibintoc, title={Referenties}]

\end{document}
