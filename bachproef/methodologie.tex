%%=============================================================================
%% Methodologie
%%=============================================================================

\chapter{\IfLanguageName{dutch}{Methodologie}{Methodology}}%
\label{ch:methodologie}

Het onderzoek verloopt over twee semesters en bestaat uit zeven fasen die samen de draft-deadline van 2 maart 2026 en de finale indiening op 29 mei 2026 haalbaar maken. Elke fase levert resultaten op die nodig zijn voor de volgende stap. Dit hoofdstuk geeft een overzicht van de gevolgde onderzoeksmethoden en de planning.

\section{Fase 0: Voorbereiding en voorstelontwikkeling}
\label{sec:fase0}

\textbf{Tijdstip:} september tot december 2025, semester 1

In semester 1 wordt het bachelorproefvoorstel voorbereid. Dit omvat het afbakenen van de onderzoeksvraag in samenspraak met co-promotor Arne Van Renthergem, het verkennen van de infrastructuur bij Mediaventures en het opstellen van een eerste versie van de literatuurstudie. Ook wordt een planning opgemaakt en worden de deliverables per fase vastgelegd. Het resultaat van deze fase is het huidige voorstel dat dient als startpunt voor het eigenlijke onderzoek in semester 2.

\section{Fase 1: Analyse van de Mediaventures-infrastructuur}
\label{sec:fase1}

\textbf{Tijdstip:} februari 2026, weken 1 tot 3

Bij aanvang van semester 2 wordt de bestaande infrastructuur bij Mediaventures in kaart gebracht. Dit gebeurt door configuratiebestanden en netwerkdiagrammen te verzamelen van de Peplink 20X- en 380X-routers, de Netgear AV-Line M4250-switches en de NDI-softwaremixers. Daarnaast worden interviews afgenomen met de technische medewerkers, waaronder co-promotor Arne Van Renthergem, om te achterhalen welke workflows gangbaar zijn en waar knelpunten zich bevinden. Tijdens een live-project wordt geobserveerd welke monitoringdata de apparaten aanbieden: statusberichten, prestatiemetingen en informatie over netwerkverkeer.

Het resultaat is een overzicht van alle databronnen die gemonitord kunnen worden en hoe deze data uit de apparaten kan worden gehaald. Dit vormt de basis voor de toolselectie in fase 3.

\section{Fase 2: Literatuurstudie}
\label{sec:fase2}

\textbf{Tijdstip:} februari tot maart 2026, weken 2 tot 5

Parallel aan de infrastructuuranalyse wordt literatuuronderzoek gedaan naar observability in audiovisuele netwerken, QoS-monitoring in streaming workflows, metrics- en logging-architecturen voor gedistribueerde systemen, en netwerksecurity- en eventanalyse. Via Google Scholar, IEEE Xplore en de ACM Digital Library worden bronnen gezocht die de termen gebruikt in dit onderzoek verder uitdiepen. Dit om ervoor te zorgen dat de gekozen aanpak wetenschappelijk onderbouwd is en dat het onderzoek begrijpbaar is voor alle lezers.

Dit onderzoek moet duidelijk maken welke tools, protocollen en architectuurmodellen relevant zijn voor dit project. Ook wordt onderzocht welke parameters volgens de literatuur het meest duiding brengen voor kwaliteitsproblemen in live-streamingomgevingen. De bevindingen worden verwerkt in de draft van 2 maart 2026.

\section{Fase 3: Vergelijkende studie van monitoringtools}
\label{sec:fase3}

\textbf{Tijdstip:} maart 2026, weken 5 tot 8

De geselecteerde monitoringtools worden vergeleken aan de hand van vijf criteria: verwerkingstijd tussen data-ingang en visualisatie, volledigheid en betrouwbaarheid van metrics, logs en flows, schaalbaarheid, integratiemogelijkheden met de Peplink-, NDI- en Netgear-apparatuur, en resourcegebruik en onderhoudscomplexiteit.

De vergelijking gebeurt aan de hand van documentatieanalyse, technische specificaties en kleinschalige tests met gesimuleerde data. Per criterium wordt elke tool beoordeeld op een vijfpuntsschaal. De resultaten worden samengevat in een vergelijkingsmatrix. Tussen 2 en 20 maart vindt ook de mondelinge toelichting aan de promotor plaats.

\section{Fase 4: Opbouw virtuele testomgeving}
\label{sec:fase4}

\textbf{Tijdstip:} maart tot april 2026, weken 7 tot 11

Omdat de productieomgeving van Mediaventures niet permanent beschikbaar is voor uitgebreide tests, wordt een virtuele testomgeving opgezet met FusionHub-instanties. Twee instanties vertegenwoordigen live-surgerylocaties, één staat voor een venue en één voor de centrale site in Bornem. Deze instanties worden gekoppeld via PepVPN-tunnels over het publieke internet, waardoor realistische WAN-condities ontstaan.

De keuze voor een volledig virtuele ontwikkelomgeving is noodzakelijk omdat testen op de live productie-infrastructuur van Mediaventures niet mogelijk is. Tijdens congressen en live-surgery projecten is de infrastructuur kritiek en mag deze niet verstoord worden door experimentele monitoring-configuraties. De observabilityoplossing wordt daarom volledig ontwikkeld en getest op basis van FusionHub-instanties.

Om te valideren in hoeverre de virtuele omgeving representatief is voor fysieke hardware, wordt aanvullend getest met een fysieke Peplink-router die ter beschikking wordt gesteld door Mediaventures. De verwachte overlap is circa 90 tot 95 procent, maar de exacte verschillen in beschikbare metrics en API-responses moeten nog worden vastgesteld. De bevindingen worden gedocumenteerd in een validatierapport dat concrete aanbevelingen bevat voor aanpassingen die nodig kunnen zijn bij een eventuele uitrol naar de productieomgeving na afloop van deze bachelorproef.

In deze omgeving worden tools ingezet die kunstmatig netwerkverkeer genereren om NDI- en SRT-videostromen na te bootsen. Deze nagebootste stromen krijgen vergelijkbare kenmerken als echt audiovisueel verkeer: wisselende datasnelheden en pieken in het verkeer. Indien mogelijk worden de Netgear-switches fysiek opgenomen in de opstelling; anders wordt hun gedrag softwarematig nagebootst.

\section{Fase 5: Connectiviteitsscenarios en bereikbaarheid van remote apparaten}
\label{sec:fase5}

Bij het opzetten van een gecentraliseerde monitoringoplossing voor gedistribueerde locaties speelt de bereikbaarheid van apparaten een fundamentele rol. Niet elke locatie beschikt over dezelfde type internetverbinding, en dit heeft directe gevolgen voor de manier waarop monitoringdata kan worden verzameld.

In een ideale situatie beschikt elke locatie over een statisch publiek IP-adres. De centrale monitoringserver kan dan rechtstreeks verbinding maken met elk apparaat om statusgegevens op te vragen. Dit model, waarbij de server actief data ophaalt bij de endpoints, wordt aangeduid als polling of pull-based monitoring \autocite{Ligus2012}.

In de praktijk is deze situatie echter eerder uitzondering dan regel. Mobiele internetverbindingen zoals 4G en 5G kennen doorgaans dynamische IP-adressen toe die regelmatig wijzigen. Bovendien bevinden apparaten zich vaak achter Network Address Translation (NAT), een techniek waarbij meerdere apparaten één publiek IP-adres delen. Apparaten achter NAT zijn niet rechtstreeks bereikbaar vanaf het internet: inkomende verbindingen worden geblokkeerd tenzij ze een antwoord zijn op een uitgaande verbinding \autocite{Srisuresh1999}.

Binnen de context van Mediaventures ontstaan hierdoor drie typische connectiviteitsscenarios:

\begin{enumerate}
\item \textbf{Beide locaties met statisch publiek IP}: Zowel de centrale site in Bornem als de remote locatie beschikken over een vast publiek IP-adres. De monitoringserver kan beide locaties rechtstreeks pollen zonder bijkomende configuratie.

\item \textbf{Eén locatie achter NAT}: De remote locatie (bijvoorbeeld een live-surgery site met 4G-verbinding) bevindt zich achter NAT en is niet rechtstreeks bereikbaar. De centrale site in Bornem heeft wel een publiek IP-adres. In dit scenario moet de remote locatie een uitgaande verbinding initiëren naar Bornem, waarna communicatie in beide richtingen mogelijk wordt.

\item \textbf{Beide locaties achter NAT}: Zowel de remote locatie als een eventuele tweede locatie (bijvoorbeeld een venue op een hotelnetwerk) bevinden zich achter NAT. Geen van beide kan de ander rechtstreeks bereiken. Een derde partij met een publiek IP-adres moet fungeren als tussenstation.
\end{enumerate}

PepVPN biedt een oplossing voor deze uitdagingen. Zodra een Peplink-router een PepVPN-tunnel opzet naar een centrale hub, ontstaat een virtueel privénetwerk waarin alle deelnemende apparaten elkaar kunnen bereiken, ongeacht hun onderliggende connectiviteitstype \autocite{Peplink2023}. De tunnel wordt geïnitieerd door de remote locatie (uitgaande verbinding), waardoor NAT-problematiek wordt omzeild. Bornem fungeert in deze architectuur als centrale hub: alle remote locaties bouwen een tunnel op naar Bornem, en via deze tunnels kan de monitoringserver alle apparaten bereiken.

Dit betekent dat de gezondheid van de PepVPN-tunnels een voorwaarde is voor effectieve monitoring. Wanneer een tunnel wegvalt, verliest de centrale server het zicht op die locatie. Het monitoren van tunnelstatus, -latency en -packet loss is daarom een essentieel onderdeel van de observabilityoplossing.

\section{Fase 6: Proof-of-concept implementatie en validatie}
\label{sec:fase6}

\textbf{Tijdstip:} april tot mei 2026, weken 10 tot 15

De gekozen monitoringoplossing wordt geïmplementeerd in de virtuele testomgeving. Dit betekent koppelingen maken met de InControl2-API en de SRT-encoders, logverzameling opzetten en dashboards bouwen die alle databronnen combineren.

De validatie gebeurt aan de hand van drie testscenario's: een ideale situatie zonder netwerkstoringen, een situatie met gesimuleerde packetlosses en vertragingspieken, en een situatie met overbelasting door meerdere gelijktijdige datastromen. Per scenario wordt gemeten hoe snel de oplossing problemen detecteert en hoe nauwkeurig de gerapporteerde metingen zijn. De resultaten worden vergeleken met de huidige werkwijze waarin techniekers handmatig moeten inloggen op elk apparaat afzonderlijk. De finale draft wordt ingediend op 4 mei 2026, waarna de laatste weken dienen voor het afronden van de bachelorproef (deadline 29 mei 2026).

\section{Planning}
\label{sec:planning}

De volledige planning van het onderzoek over twee semesters wordt weergegeven in het Gantt-diagram in figuur~\ref{fig:gantt-methodologie}. Het diagram toont de tijdslijn vanaf de voorbereidingsfase in semester 1 tot en met de definitieve indiening in semester 2, inclusief alle belangrijke mijlpalen en deadlines.

\begin{figure}[h]
\centering
\includegraphics[width=\linewidth,height=0.3\textheight,keepaspectratio]{mermaid_bap.png}
\caption{Gantt-diagram van de onderzoeksfasen over twee semesters.}
\label{fig:gantt-methodologie}
\end{figure}

\section{Risico's en reflectie}
\label{sec:risicos}

Het onderzoek brengt verschillende risico's met zich mee:

\textbf{Beschikbaarheid testinfrastructuur}: De productieomgeving van Mediaventures is niet permanent beschikbaar voor uitgebreide tests. Dit wordt opgelost door een virtuele testomgeving met FusionHub op te zetten, waardoor reproduceerbare tests mogelijk zijn zonder afhankelijkheid van fysieke apparatuur tijdens live-projecten.

\textbf{Complexiteit multi-vendor integratie}: De infrastructuur combineert componenten van verschillende leveranciers (Peplink, Netgear, diverse NDI- en SRT-apparaten) met elk hun eigen managementprotocollen. In fase 1 wordt daarom uitgebreid onderzocht welke API's en exportmogelijkheden beschikbaar zijn. Wanneer bepaalde componenten geen standaard exportformaten ondersteunen, kunnen custom exporters worden ontwikkeld of alternatieve dataverzamelingsmethoden worden toegepast.

\textbf{Beperkte toegang tot vergelijkbare onderzoekscontexten}: Audiovisuele livestreaming voor medische congressen is een nichemarkt met weinig publieke documentatie. Dit wordt opgevangen door nauwe samenwerking met co-promotor Arne Van Renthergem van Mediaventures, die praktijkervaring en inzicht in specifieke uitdagingen kan delen. Daarnaast wordt de literatuurstudie breder getrokken naar aanverwante domeinen zoals IP-video surveillance, live broadcasting en enterprise network monitoring.

\textbf{Tijdsmanagement}: Het onderzoek combineert theorie en praktische implementatie binnen een beperkt tijdsbestek. De planning is daarom opgedeeld in duidelijke fasen met concrete mijlpalen. Regelmatige voortgangsbesprekingen met de promotor en co-promotor zorgen voor tijdige feedback en bijsturing. De keuze voor open-source tools met actieve communities verkleint het risico op langdurige technische blokkades.

