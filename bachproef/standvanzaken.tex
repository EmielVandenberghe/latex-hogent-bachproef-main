\chapter{\IfLanguageName{dutch}{Stand van zaken}{State of the art}}%
\label{ch:stand-van-zaken}

Dit hoofdstuk geeft een overzicht van de huidige stand van zaken binnen het onderzoeksdomein. Het beschrijft de technologieën die gebruikt worden in de audiovisuele infrastructuur van Mediaventures en de uitdagingen op het vlak van monitoring en observability.

\section{Observability in audiovisuele netwerkarchitecturen}
\label{sec:observability-av}

In audiovisuele productieomgevingen is observability een essentieel concept om de betrouwbaarheid en prestaties van complexe ketens te kunnen beoordelen. Videostreams zijn gevoelig voor vertraging en verlies van datapakketten, waardoor afwijkingen in de netwerk-, transport- of applicatielaag onmiddellijk zichtbare impact hebben op de uiteindelijke weergave. Dit geldt in het bijzonder voor live-omgevingen waarin meerdere geografisch gescheiden locaties gelijktijdig video en audio uitwisselen. \textcite{Tommasi2021} tonen aan dat real-time monitoring van parameters zoals latency en packet loss cruciaal is om degradatie in streamingkwaliteit tijdig te detecteren. De literatuur benadrukt daarbij dat observability niet beperkt mag blijven tot netwerkmetingen, maar juist die verschillende niveaus moet combineren (netwerklaag, transportlaag en applicatielaag) om complexe audiovisuele workflows correct te interpreteren.

\section{NDI als intern audiovisueel transportprotocol}
\label{sec:ndi}

NDI (Network Device Interface) is een AV-over-IP-technologie die ontworpen is voor situaties waarin meerdere videobronnen binnen een lokaal netwerk (LAN) beschikbaar moeten zijn. Volgens de officiële NDI-documentatie maakt het protocol gebruik van hoge bandbreedtes en lage latency om videobronnen vrijwel in realtime tussen apparaten te transporteren \autocite{NewTek2022}. NDI-apparaten kunnen elkaar automatisch ontdekken binnen hetzelfde subnet, waardoor videobronnen flexibel kunnen worden gerouteerd zonder bijkomende bekabeling.

Binnen Mediaventures ontvangt een computer de verschillende NDI-streams en verwerkt deze via een softwarematige mixer tot één gecombineerd eindbeeld \autocite{VanRenthergem2025}. Dit mixstation vormt een kritisch punt in de workflow: omdat NDI-prestaties afhankelijk zijn van bandbreedte, multicastondersteuning en correcte VLAN-configuratie \autocite{NewTek2022}, heeft elke afwijking in de netwerkcondities of processorbelasting direct impact op het eindbeeld dat vervolgens via Secure Reliable Transport (SRT) wordt verzonden naar de ontvangende locatie.

\section{SRT voor transport over grotere afstanden}
\label{sec:srt}

\subsection{De omzetting van NDI naar SRT}

Omdat NDI ontworpen is voor gebruik binnen een LAN en niet geschikt is voor transmissie over grote afstanden, wordt de samengestelde NDI-output dus lokaal omgezet naar SRT (Secure Reliable Transport). Haivision, de organisatie achter SRT, beschrijft het protocol als een oplossing die storingen in internetverbindingen compenseert door middel van foutcorrectie en jitterbuffers \autocite{SRTspec2020}. Tijdens dit omzettingsproces wordt de NDI-stream gecodeerd naar een geschikt videocompressieformaat en verpakt in een SRT-stream.

Deze omzetting vormt een belangrijke overgang van LAN-gebaseerde videodistributie naar WAN-transmissie. Storingen in de NDI-laag (zoals haperingen of te hoge belasting) worden op dit punt direct doorgegeven aan de SRT-encoder, die ze niet altijd kan corrigeren. Dit maakt monitoring op deze overgang van bijzonder belang \autocite{VanRenthergem2025}.

\subsection{De omzetting van SRT terug naar een lokaal formaat}

Wanneer de SRT-stream aankomt op de venue of in Bornem (de centrale site), wordt deze opnieuw gedecodeerd. De ontvangende software zet de stream terug om in een videobron die binnen het lokale netwerk kan worden gebruikt. Volgens \textcite{SRTspec2020} ondersteunt SRT dit decodeerproces zonder dat de oorspronkelijke kwaliteit onnodig wordt aangetast. In veel audiovisuele workflows wordt de gedecodeerde stream opnieuw als NDI beschikbaar gemaakt zodat deze lokaal op dezelfde manier kan worden gerouteerd als andere videobronnen. Dit zorgt voor een consistente end-to-end workflow waarin NDI wordt gebruikt voor interne distributie en SRT voor transmissie over grotere afstanden \autocite{VanRenthergem2025}.

\subsection{SRT-statistieken en monitoring van streamkwaliteit}

Het SRT-protocol biedt uitgebreide mogelijkheden voor het monitoren van de verbindingskwaliteit in real-time. De protocolspecificatie definieert een reeks statistieken die zowel aan zenderzijde als aan ontvangerzijde beschikbaar zijn en inzicht geven in de gezondheid van de verbinding \autocite{SRTAlliance2024}.

De belangrijkste parameters voor het beoordelen van de streamkwaliteit zijn:

\begin{figure}[htbp]
  \centering
  \includegraphics[width=\linewidth,height=0.38\textheight,keepaspectratio]{chatgpt_srt_explanation.png}
  \caption[NDI-naar-SRT-naar-NDI workflow]{Schematische weergave van de volledige AV-workflow: camera's op de surgery-locatie sturen via NDI naar een mixer-pc en SRT-encoder, waarna de stream via het internet naar de venue wordt getransporteerd en er opnieuw als NDI beschikbaar wordt gemaakt \autocite{ChatGPT2026}.}
  \label{fig:srt-workflow}
\end{figure}

\begin{itemize}
\item \textbf{Round-Trip Time (RTT)}: De tijd in milliseconden die een pakket nodig heeft om van zender naar ontvanger en terug te reizen. Deze waarde is cruciaal voor het berekenen van de optimale latency-instelling. De aanbevolen SRT-latency is minimaal drie tot vier keer de gemiddelde RTT \autocite{Sharabayko2022}.

\item \textbf{Packet loss}: Het percentage pakketten dat verloren gaat tijdens transport. SRT compenseert pakketverlies via Automatic Repeat Query (ARQ), waarbij verloren pakketten opnieuw worden verzonden. Bij te hoge packet loss of te lage latency-instellingen kunnen pakketten niet tijdig worden hersteld, wat resulteert in zichtbare artefacten in het videobeeld \autocite{SRTspec2020}.

\item \textbf{Jitter}: De variatie in aankomsttijd van pakketten. SRT gebruikt een jitterbuffer om deze variatie op te vangen en een constante end-to-end latency te garanderen. Hoge jitter vereist een grotere buffer en dus hogere latency om pakketverlies te voorkomen.

\item \textbf{Retransmissions}: Het aantal pakketten dat opnieuw is verzonden na initieel verlies. Een hoog aantal retransmissies duidt op een instabiele verbinding en verhoogt de effectieve bandbreedte die nodig is voor de stream.

\end{itemize}

Deze statistieken worden door SRT-encoders en -decoders beschikbaar gesteld via interne API's of management-interfaces. Voor monitoring doeleinden kunnen deze waarden periodiek worden uitgelezen en geëxporteerd naar een centraal observability-platform. De correlatie tussen SRT-statistieken en onderliggende WAN-kwaliteit (zoals gemeten via PepVPN) maakt het mogelijk om de oorzaak van streamingproblemen sneller te identificeren: een stijging in RTT of packet loss op de SRT-laag kan direct worden gekoppeld aan degradatie van een specifieke WAN-verbinding.

\section{Peplink, PepVPN en de rol van multi-WAN-routing}
\label{sec:peplink}

De netwerkarchitectuur tussen de verschillende locaties wordt gerealiseerd met Peplink-routers die PepVPN gebruiken. Deze technologie creëert één logische VPN-tunnel die gelijktijdig over meerdere fysieke internetverbindingen (zoals 5G, Starlink of glasvezel) loopt \autocite{Peplink2023}. Omdat de tunnel gebruik maakt van alle beschikbare verbindingen tegelijk, blijft deze operationeel wanneer een link uitvalt: het verkeer wordt dan automatisch over de overgebleven verbindingen herverdeeld zonder dat de VPN-sessie verbroken wordt.

De live-surgery locaties zijn uitgerust met Peplink 20X-routers die via USB-modems toegang hebben tot 5G-netwerken. De venues beschikken over krachtigere Peplink 380X-routers die hogere bandbreedtes kunnen verwerken. Ook in de centrale site in Bornem staat een Peplink-router die fungeert als tussenknooppunt wanneer verkeer niet rechtstreeks tussen locaties wordt gerouteerd \autocite{VanRenthergem2025}. Deze redundantie is cruciaal voor live-productieomgevingen waarin onderbrekingen direct merkbaar zijn.

\begin{figure}[htbp]
  \centering
  \includegraphics[width=0.82\linewidth,height=0.40\textheight,keepaspectratio]{chatgpt_topology_mediav.png}
  \caption[Netwerktopologie van de Mediaventures-infrastructuur]{Schematische weergave van de Peplink-netwerktopologie bij Mediaventures: Bornem fungeert als centrale hub (380X) en is via PepVPN-tunnels verbonden met twee surgery-locaties (20X) en een venue (380X) \autocite{ChatGPT2026}.}
  \label{fig:topology-mediav}
\end{figure}

\section{InControl2 als platform voor routerobservatie}
\label{sec:incontrol2}

InControl2 is het cloudplatform waarmee Peplink beheerders toelaat om routers centraal te monitoren. Volgens de officiële documentatie biedt het platform inzicht in WAN-status, historische latenties, VPN-kwaliteit en configuratiebeheer \autocite{Peplink2023}. Hoewel InControl2 waardevolle routerinformatie biedt, vormt het geen complete observabilityoplossing. Het platform monitort geen audiovisuele protocollen zoals NDI en SRT, en integreert evenmin gegevens van andere netwerkcomponenten. Deze beperking is inherent aan het single-vendor karakter van het platform: omdat InControl2 uitsluitend Peplink-apparatuur bewaakt, ontbreekt elk inzicht in de toestand van andere componenten in de keten.

\section{Virtuele onderzoekomgeving via FusionHub}
\label{sec:fusionhub}

FusionHub is de virtuele router van Peplink, beschikbaar als VM-image voor deployment in cloudomgevingen zoals AWS of Google Cloud, maar ook lokaal via hypervisors zoals VirtualBox of VMware. FusionHub-instanties registreren zich bij InControl2, ondersteunen PepVPN-tunnels en bieden toegang tot de InControl2 API \autocite{Peplink2023}.

Volgens \textcite{VanRenthergem2025} is de functionele overlap tussen FusionHub en fysieke Peplink-hardware circa 90 tot 95 procent. Verschillen bestaan voornamelijk in hardware-specifieke features zoals cellular modem statistics en fysieke WAN-interfaces, die op FusionHub niet aanwezig zijn.

\subsection{Keuze van observability-tooling}

Voor het verzamelen, opslaan en visualiseren van observability-data bestaan verschillende open-source en commerciële oplossingen. In deze sectie wordt de keuze voor de gebruikte toolstack onderbouwd.

De meest gangbare open-source monitoring stacks zijn gebaseerd op Prometheus voor metrics-verzameling, Grafana voor visualisatie en Loki voor log-aggregatie. Deze combinatie, vaak aangeduid als de PLG-stack, wordt breed ingezet in cloud-native omgevingen en DevOps-contexten \autocite{Turnbull2018}. Prometheus werkt volgens een pull-model waarbij de server periodiek endpoints bevraagt, en slaat data op in een geoptimaliseerde time-series database. Grafana biedt een flexibele interface voor het bouwen van dashboards en ondersteunt meerdere databronnen, waaronder Prometheus en Loki.

Alternatieven die werden overwogen zijn Zabbix, Nagios en commerciële platformen zoals Datadog en Splunk. Zabbix en Nagios zijn traditionele monitoringoplossingen met een lange staat van dienst in enterprise-omgevingen \autocite{Turnbull2016}. Deze tools zijn echter primair ontworpen voor server- en infrastructuurmonitoring en bieden minder flexibiliteit voor het integreren van custom metrics uit diverse bronnen zoals de InControl2 API of SRT-encoders. Commerciële oplossingen zoals Datadog bieden uitgebreide functionaliteit maar brengen aanzienlijke licentiekosten met zich mee, wat ze minder geschikt maakt voor een proof-of-concept in een bachelorproefcontext.

De keuze voor de PLG-stack is gebaseerd op verschillende overwegingen. Ten eerste biedt Prometheus ondersteuning voor custom exporters, waardoor data uit de InControl2 API en SRT-statistieken eenvoudig kan worden geïntegreerd. Daarnaast zijn alle componenten open-source en kunnen ze zonder licentiekosten worden ingezet, wat belangrijk is voor een proof-of-concept in een bachelorproefcontext. De uitgebreide documentatie en actieve community verlagen de drempel voor implementatie. Tot slot is de architectuur geschikt voor zowel kleinschalige proof-of-concepts als grotere productieomgevingen, en wordt de PLG-stack breed toegepast in vergelijkbare observability-projecten, wat overdraagbaarheid en herkenbaarheid ten goede komt.

\begin{figure}[htbp]
  \centering
  \includegraphics[width=0.62\linewidth,height=0.52\textheight,keepaspectratio]{chatgpt_plg_stack.png}
  \caption[Architectuur van de PLG-stack]{Architectuur van de PLG-stack voor dit onderzoek: de InControl2 API en SRT-encoderstatistieken worden via een Python custom exporter omgezet naar het Prometheus-formaat, opgeslagen in de Prometheus time-series database en gevisualiseerd via Grafana \autocite{ChatGPT2026}.}
  \label{fig:plg-stack}
\end{figure}

Voor dit onderzoek wordt daarom gekozen voor Prometheus als metrics-backend, Grafana als visualisatieplatform en Loki voor eventuele log-aggregatie. Custom Python exporters worden ontwikkeld om data uit de InControl2 API en SRT-encoders om te zetten naar het Prometheus-formaat.

\subsection{Synthese}
\label{sec:synthese-standvanzaken}

De literatuur toont dat audiovisuele workflows meerdere gespecialiseerde protocollen, netwerkcomponenten en afhankelijkheden combineren. NDI presteert optimaal binnen LAN-omgevingen voor lokale videodistributie, terwijl SRT ontworpen is voor betrouwbaar WAN-transport over onvoorspelbare internetverbindingen. Peplink's PepVPN biedt de netwerklaag die deze locaties verbindt via multi-WAN bonding over 5G, Starlink en glasvezel.

Deze drie lagen zijn onlosmakelijk met elkaar verbonden: problemen in de WAN-laag (zoals packet loss op een 5G-verbinding) manifesteren zich direct in de SRT-statistieken (verhoogde RTT en retransmissies), wat uiteindelijk zichtbaar wordt als kwaliteitsverlies in de NDI-output op de venue. Omgekeerd kan overbelasting van de lokale NDI-mixer leiden tot verhoogde jitter in de SRT-stream, zelfs wanneer de onderliggende WAN-verbinding stabiel is. Zonder inzicht in alle lagen tegelijk is het voor techniekers moeilijk om de werkelijke oorzaak van problemen te achterhalen.

Geen enkel bestaand platform monitort al deze lagen tegelijk. InControl2 biedt waardevolle WAN-metrics maar mist SRT- en NDI-inzichten. SRT-encoders rapporteren streamstatistieken maar kennen de toestand van de onderliggende netwerk\-infrastructuur niet. De observability-oplossing die in dit onderzoek wordt ontwikkeld, combineert daarom SRT-statistieken van de encoders met WAN-kwaliteitsmetrics uit de InControl2 API in één gecentraliseerd platform. Door deze databronnen te correleren in Grafana-dashboards kunnen techniekers in één oogopslag zien of een streamingprobleem veroorzaakt wordt door lokale netwerkbelasting, WAN-degradatie of encoder-issues.

Dit onderzoek start met een afgebakende scope gericht op SRT en PepVPN-verbindingen. Wanneer deze kernprotocollen succesvol gemonitord worden en de tijd het toelaat, kunnen aanvullende componenten zoals NDI-statistieken en switch-monitoring geleidelijk worden geïntegreerd. De bedoeling is om ook na afloop van deze bachelorproef verder te werken aan een alomvattende monitoringoplossing voor Mediaventures.

