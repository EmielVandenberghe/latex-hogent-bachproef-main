%%=============================================================================
%% Samenvatting
%%=============================================================================

% TODO: De "abstract" of samenvatting is een kernachtige (~ 1 blz. voor een
% thesis) synthese van het document.
%
% Een goede abstract biedt een kernachtig antwoord op volgende vragen:
%
% 1. Waarover gaat de bachelorproef?
% 2. Waarom heb je er over geschreven?
% 3. Hoe heb je het onderzoek uitgevoerd?
% 4. Wat waren de resultaten? Wat blijkt uit je onderzoek?
% 5. Wat betekenen je resultaten? Wat is de relevantie voor het werkveld?
%
% Daarom bestaat een abstract uit volgende componenten:
%
% - inleiding + kaderen thema
% - probleemstelling
% - (centrale) onderzoeksvraag
% - onderzoeksdoelstelling
% - methodologie
% - resultaten (beperk tot de belangrijkste, relevant voor de onderzoeksvraag)
% - conclusies, aanbevelingen, beperkingen
%
% LET OP! Een samenvatting is GEEN voorwoord!

%%---------- Nederlandse samenvatting -----------------------------------------
%
% TODO: Als je je bachelorproef in het Engels schrijft, moet je eerst een
% Nederlandse samenvatting invoegen. Haal daarvoor onderstaande code uit
% commentaar.
% Wie zijn bachelorproef in het Nederlands schrijft, kan dit negeren, de inhoud
% wordt niet in het document ingevoegd.

\IfLanguageName{english}{%
\selectlanguage{dutch}
\chapter*{Samenvatting}
% Nederlandse samenvatting voor Engelstalige thesis (niet van toepassing)
\selectlanguage{english}
}{}

%%---------- Samenvatting -----------------------------------------------------
% De samenvatting in de hoofdtaal van het document

\chapter*{\IfLanguageName{dutch}{Samenvatting}{Abstract}}

Mediaventures is een audiovisuele firma die gespecialiseerd is in livestreamingsdiensten, zaalbouw en -inrichting voor internationale medische congressen zoals het LINC Congress en het ESC Congress. Tijdens hun projecten worden verschillende live-surgery-locaties en venues met een centrale site verbonden om hoogwaardige video- en audiostreams in realtime uit te wisselen. Deze samenwerking steunt op een complexe mix van netwerkapparatuur en verbindingen, waardoor de kwaliteit van de livestream door veel verschillende factoren tegelijk beïnvloed wordt. Vandaag bestaat er echter geen centraal systeem dat deze volledige keten bewaakt en vroegtijdig aangeeft wanneer er problemen ontstaan. Daardoor moeten techniekers op elk toestel afzonderlijk inloggen om fouten op te sporen, wat tijdrovend is en het risico op kwaliteitsverlies verhoogt.

Dit onderzoek onderzoekt hoe een geïntegreerde monitoringsoplossing kan helpen om deze diverse infrastructuur end-to-end op te volgen en sneller inzicht te krijgen in de gezondheid en prestaties van alle betrokken systemen. De focus ligt daarbij op detectie en zichtbaarmaking van problemen via een gecentraliseerd dashboard.
