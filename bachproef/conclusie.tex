%%=============================================================================
%% Conclusie
%%=============================================================================

\chapter{Conclusie}%
\label{ch:conclusie}

% TODO: Dit hoofdstuk bevat momenteel de verwachte resultaten uit het voorstel.
% Na afronding van het onderzoek moeten deze vervangen worden door de werkelijke
% conclusies en antwoorden op de onderzoeksvragen.

Dit onderzoek zal resulteren in een onderbouwde aanbeveling voor een monitoringoplossing die geschikt is voor de audiovisuele infrastructuur van Mediaventures. De toolvergelijking zal duidelijk maken welke combinatie van monitoring- en observabilitysoftware het best aansluit bij de specifieke eisen van live-surgeryprojecten en medische congressen. De aanbeveling zal gebaseerd zijn op concrete testresultaten uit de virtuele omgeving en de vergelijking met bestaande oplossingen uit de literatuur.

Het proof-of-concept zal naar verwachting aantonen dat gecentraliseerde monitoring de tijd tussen het ontstaan van een probleem en de detectie ervan aanzienlijk verkort. Waar techniekers momenteel handmatig moeten inloggen op afzonderlijke apparaten om relevante statusinformatie te verzamelen, zal een gecentraliseerd monitoringplatform alle beschikbare metingen, logbestanden en netwerkgegevens gebundeld presenteren via dashboards. Concreet wordt verwacht dat de detectietijd voor veelvoorkomende problemen zoals packetloss, vertragingspieken en netwerkoverbelasting opmerkelijk afneemt ten opzichte van de huidige werkwijze.

Het onderzoek zal ook duidelijk maken welke meetwaarden het belangrijkst zijn om kwaliteitsproblemen te detecteren. Deze inzichten kunnen techniekers helpen om sneller en gerichter te reageren op problemen.

Tot slot wordt verwacht dat de virtuele testomgeving met FusionHub een reproduceerbare basis biedt voor verdere ontwikkeling. De configuraties en dashboards die tijdens het onderzoek worden ontwikkeld, kunnen na afloop worden overgedragen aan Mediaventures voor integratie in hun dagelijkse werking. De meerwaarde voor het bedrijf bestaat dus niet alleen uit een aanbeveling, maar ook uit een werkend prototype dat als startpunt dient voor verdere uitbouw.

